\documentclass[10pt,conference]{IEEEtran}
\IEEEoverridecommandlockouts
% The preceding line is only needed to identify funding in the first footnote. If that is unneeded, please comment it out.
\usepackage{cite}
\usepackage{amsmath,amssymb,amsfonts}
\usepackage{algorithmic}
\usepackage{graphicx}
\usepackage{textcomp}
\usepackage{xcolor}
\usepackage{xspace}
\usepackage{fontawesome}
\usepackage{multirow}
\usepackage{rotating}
\usepackage{tikz}
\usepackage{soul}
\usepackage{makecell}
\usepackage{url}
\usepackage{enumitem}
\usepackage{booktabs}
\usepackage{fontawesome}
\usepackage{amsmath}
\usepackage{graphicx}
\usepackage[colorlinks=true, allcolors=blue]{hyperref}
\usepackage{pifont}
\usepackage{multicol}
\usepackage{multirow}

\usepackage{amssymb}% http://ctan.org/pkg/amssymb
\usepackage{pifont}% http://ctan.org/pkg/pifont
\newcommand{\cmark}{\ding{51}}%
\newcommand{\xmark}{\ding{55}}%

\newcommand\TODO[1]{\textcolor{red}{#1}}
\newcommand\REMOVE[1]{\textcolor{red}{\st{#1}}}
\newcommand\ANTONIO[1]{\textcolor{blue}{{ANTONIO}{#1}}}
\newcommand\LUCA[1]{\textcolor{green}{{LUCA}{#1}}}
\newcommand\GABRIELE[1]{\textcolor{red}{{GABRIELE}{#1}}}
\newcommand{\ie}{\emph{i.e.,}\xspace}
\newcommand{\eg}{\emph{e.g.,}\xspace}
\newcommand{\etc}{etc.\xspace}
\newcommand{\etal}{\emph{et~al.}\xspace}
\newcommand{\secref}[1]{Section~\ref{#1}\xspace}
\newcommand{\chapref}[1]{Chapter~\ref{#1}\xspace}
\newcommand{\appref}[1]{Appendix~\ref{#1}\xspace}
\newcommand{\figref}[1]{Fig.~\ref{#1}\xspace}
\newcommand{\listref}[1]{Listing~\ref{#1}\xspace}
\newcommand{\tabref}[1]{Table~\ref{#1}\xspace}
\newcommand{\tool}[1]{{\sc #1}\xspace}
\newcommand{\highlight}[1]{{\color{purple} \textbf{#1}}}
\newcommand{\approach}{LEONID\xspace} %Log gEneratiON using Information retrieval and Deep learning
\newcommand{\java}{\emph{Java}\xspace}

	
\newcommand*\circled[1]{\tikz[baseline=(char.base)]{
		\node[shape=circle,draw,inner sep=1pt] (char) {#1};}}
		
		
\def\BibTeX{{\rm B\kern-.05em{\sc i\kern-.025em b}\kern-.08em
    T\kern-.1667em\lower.7ex\hbox{E}\kern-.125emX}}


\begin{document}

\title{Log Statements Generation via Deep Learning:\\Widening the Support Provided to Developers\vspace{-1cm}}

%\author{
%\IEEEauthorblockN{Antonio Mastropaolo, Luca Pascarella, Gabriele Bavota}
%\IEEEauthorblockA{\textit{Software Institute -- USI Universit\`{a} della Svizzera italiana, Switzerland}}
%}

\maketitle

\begin{abstract}
Logging helps keeping track of events occurring during software execution. Previous work stressed the challenges developers face when logging. These include: \emph{where} to log, \emph{what} information to log, and which \emph{log level} to use (\eg info, fatal). To support developers in all these decisions, Mastropaolo \etal presented LANCE, deep learning (DL)-based approach to automatically inject a log statement in a given Java method. While being the state-of-the-art in the automation of logging activities, LANCE suffers of two major limitations. First, given a method as input, LANCE always assumes that it requires the injection of exactly \emph{one} new log statement, even in cases in which \emph{no} additional log statements would be needed or \emph{multiple} log statement would be required. This is due to the specific training procedure put in place by Mastropaolo \etal Second, LANCE is only able to generate meaningful log messages in 15.2\% of cases. In this work, we study how to address these limitations to widen the support provided to developers in terms of logging automation. We start by replicating LANCE on a dataset 3.6 times larger than the one used by Mastropaolo \etal Then, we present \approach, an extension of LANCE which is able to (i) decide whether a given Java method would actually benefit from the injection of log statements (or if, instead, those are not needed); and (ii) support the injection of multiple log statements if needed, deciding how many statements are needed, where they should be placed, and what they should log. Finally, we also tried to boost the generation of meaningful log messages by combining information retrieval (IR) and DL. Our results show that: (i) \approach can accurately discriminate between methods needing and not needing the addition of log statements; (ii) \approach can support, when needed, the injection of multiple log statements in a given method, without a substantial loss in performance as compared to the simpler single-log injection problem addressed by LANCE; (iii) by just increasing the size of the training dataset, LANCE's ability to predict meaningful log messages improves by a factor of two, approaching 30\%; (iv) \approach can only slightly improve the performance of LANCE in generating meaningful log messages thanks to the combination of DL and IR, thus calling for additional research on this specific problem.
\end{abstract}

\begin{IEEEkeywords}
Logging, DL for Software Engineering
\end{IEEEkeywords}

% !TEX root = main.tex
%%%%%%%%%%%%%%%%%%%%%%%%%%%%%%%%%%%%%%%%
%%%%%%%%%%%%%%%%%%%%%%%%%%%%%%%%%%%%%%%%
\section{Introduction} \label{sec:intro}
%%%%%%%%%%%%%%%%%%%%%%%%%%%%%%%%%%%%%%%%
%%%%%%%%%%%%%%%%%%%%%%%%%%%%%%%%%%%%%%%%
% !TEX root = main.tex
%%%%%%%%%%%%%%%%%%%%%%%%%%%%%%%%%%%%%%%%
%%%%%%%%%%%%%%%%%%%%%%%%%%%%%%%%%%%%%%%%
\section{\approach} \label{sec:t5}
%%%%%%%%%%%%%%%%%%%%%%%%%%%%%%%%%%%%%%%%
%%%%%%%%%%%%%%%%%%%%%%%%%%%%%%%%%%%%%%%%

We start by providing an introduction to the T5 model we use (\secref{sub:t5}). Then, we describe how we built the datasets used for the different training phases we dealt with (\secref{sub:datasets}).  \secref{sec:training} will then explain how we used these datasets to run the actual training process.


\subsection{Text-to-Text-Transfer-Transformer (T5)}
\label{sub:t5}
T5 has been introduced by Raffel \etal \cite{raffel2019exploring} as a Transformer \cite{vaswani2017attention} model to support multitask learning in the domain of NLP (Natural Language Processing). The idea behind the T5 model is to reframe NLP tasks in a unified text-to-text format in which the input and output of the model are text strings. This implies that a single model can be trained to translate across languages (\eg from English to Spanish) and to autocomplete sentences. The training of T5 includes two phases. The first is the \textit{pre-training}, in which the model is trained with a self-supervised objective to acquire general knowledge about the language(s) of interest. In our example, this may mean providing as input to the model English sentences having a subset of their words masked and asking the model to generate as output the masked words. Being self-supervised (\ie the training instances can be automatically generated by masking random words) the pre-training can usually be performed on large-scale datasets. Once pre-trained, T5 can be fine-tuned to support specific tasks with supervised training objectives. This means, for example, providing it with pairs of sentences $<$\emph{english}, \emph{spanish}$>$.

In our work, we rely on the same T5 architecture (\ie T5$_{small}$) that has been exploit by Mastropaolo \etal \cite{mastropaolo2022using} in LANCE. T5\textsubscript{\textit{small}} is characterized by six blocks for encoders and decoders. The feed-forward networks in each block consist of a dense layer with an output dimensionality ($d_{ff}$) of 2,048. The \textit{key} and \textit{value} matrices of all attention mechanisms have an inner dimensionality ($d_{kv}$) of 64, and all attention mechanisms have eight heads. All the other sub-layers and embeddings have a dimensionality ($d_{model}$) of 512. The code implementing T5 is available in our replication package \cite{replication}.


\subsection{Datasets Needed for Training, Validation, and Testing}
\label{sub:datasets}

We start by describing the dataset used for pre-training T5 (\secref{sub:pretraining}). Then, we detail the several fine-tuning datasets we built (featuring training, validation, and test set). The first, aimed at replicating LANCE \cite{mastropaolo2022using}, teaches T5 how to inject a single log statement in a Java method  (\secref{sec:single-log-dataset}). The second fine-tuning dataset also focuses on the problem of injecting a single log statement, but this time exploits IR to provide T5 with concrete examples of log messages that might be relevant for the prediction at hand (\secref{sec:single-log-plus-IR}). This allows to directly compare LANCE with \approach in the task of single log statement generation. The third fine-tuning dataset trains \approach for the task of multi-log statements prediction, \ie the ability to inject from 1 to $n$ log statements in a given method (\secref{sec:multi-log-dataset}). Also this fine-tuning dataset exploits IR to boost the prediction accuracy. Finally, we show how we built a fine-tuning dataset to tech T5 how to discriminate between methods needing and not needing log statements (\secref{sec:predicting-dataset}). The datasets are summarized in \tabref{tab:ds-summary-1} and \tabref{tab:ds-summary-2}.



All datasets have been built starting from the same set of GitHub repositories that we selected using the GHS (GitHub Search) tool by Dabi\'c \etal \cite{dabic2021sampling}. GHS allows to query GitHub for projects meeting specific criteria. We used the same selection criteria by Mastropaolo \etal \cite{mastropaolo2022using}, selecting all public non-forked \java projects having at least 500 commits, 10 contributors, and 10 stars. These selection criteria aim at excluding personal/toy projects and reduce the chance of collecting duplicated code (non-forked repositories). We cloned the latest snapshot of the 6,352 projects returned by GHS. We scanned all cloned repositories to assess whether they featured a \texttt{POM} (Project Object Model) or a \texttt{build.gradle} file. Both these files allow to declare external dependencies towards libraries, the former using Maven, the latter Gradle. Such a check was performed since, as a subsequent step, we verify whether projects had a dependency towards Apache Log4j \cite{log4j} (\ie a well-known Java logging library) or SLF4J (Simple Logging Facade for Java) \cite{slf4j} (\ie an abstraction for Java logging frameworks similar to Log4j). Indeed, to train a T5 for the task of injecting complete log statement(s) in Java methods, we needed examples of methods featuring log statements. The usage of popular logging Java libraries was thus a prerequisite for the project's selection.

We found 3,865 projects having either a \texttt{POM} or a \texttt{build.gradle} file and 2,978 of them featured a dependency towards the two targeted logging libraries. The overall projects' selection is very similar to the one by Mastropaolo \etal \cite{mastropaolo2022using}, with the main differences being the additional mining of projects: (i) using Gradle as build system (in \cite{mastropaolo2022using} only Maven was considered); and (ii) having a dependency towards SLF4J (in \cite{mastropaolo2022using} only Log4j was considered as logging library). These choices help in increase the size and variety of both the training and the testing datasets, making the prediction more challenging. 

We then used srcML \cite{srcml} to perform a lightweight parsing of the selected projects. First, we extracted all \java methods they feature. Then, we identified the log statements within each method (if any) and removed all methods featuring log statements exploiting custom log levels (\ie log levels that do not belong to any of the two libraries we consider in our study but that have been defined within a specific project). The valid log levels we considered are: \texttt{FATAL}, \texttt{ERROR}, \texttt{WARN}, \texttt{DEBUG}, \texttt{INFO}, and \texttt{TRACE}. At this point we were left with two sets of methods: those not having any log statement and those having at least one log statement using one of the ``valid'' log levels.

We run \emph{javalang} \cite{javalang} on the remaining methods to tokenize them and excluded all those having $\#tokens < 10$ or $\#tokens \geq 512$. The upper-bound filtering has been done in previous works \cite{ADD_CITATIONS_FROM_OTHER_GROUPS,mastropaolo2021empirical,tufano2021automating,ciniselli2021empirical} to limit the computational expenses of training DL-based models. The lower-bound of 10 tokens aims at removing empty methods. We also filtered out all the methods containing non-ASCII characters in an attempt to exclude at least some of the methods featuring log messages not written in English. 


Finally, to avoid any possible overlap between the training, evaluation, and test datasets we are going to create from the collected set of methods, we removed all exact duplicates, obtaining the final set of 12,917,300 \java methods, of which 244,588 contain at least one log statement. 

\begin{table*}[h]
	\centering
	\caption{Number of methods in the datasets used in our study}
		\label{tab:ds-summary-1}
	\begin{tabular}{ccccccccc}
		\toprule
		\multirow{2}{*}{\textit{\textbf{Dataset}}} & \multicolumn{2}{c}{\textbf{train}} & \textbf{} & \textbf{eval} & \textbf{} & \textbf{test}  \\ \cline{2-3} \cline{5-5} \cline{7-7} 
		& \textbf{w/ log} & \textbf{w/o log} & \textbf{} & \textbf{w/ log} & \textbf{} & \textbf{w/ log} \\ \midrule
		\textit{Pre-training}              & -               &      12,671,475  &           & -               &           &  -               \\
		\textit{Fine-tuning: Single Log Generation}               & 229,703         & -                &           & 28,763          &           & 28,698          \\
		\textit{Fine-tuning: Single Log Generation with IR}               & 229,703         & -                &           & 28,763          &           & 28,698          \\
		\textit{Fine-tuning: Multi-log Injection with IR}               & 192,773         & -                &           & 24,092         &           & 24,088          \\
		\bottomrule
	\end{tabular}
\end{table*}

\subsubsection{Pre-Training Dataset}
\label{sub:pretraining}
Since the goal of pre-training is to provide T5 with general knowledge about the language of interest (\ie Java), we used for pre-training all methods not featuring a log statement (the latter will be used for the fine-tuning datasets). For the pre-training we adopted a classic \emph{masked language model} task, which consists in randomly masking 15\% of the tokens composing a training instance (\ie a Java method in our case) asking the model to predict them. \figref{fig:pre-training} depicts a pre-training instance from our dataset.

\begin{figure}[h!]
	\label{fig:pre-training}
	\includegraphics[width=\columnwidth]{img/pre-training.pdf}
		\caption{Example of Pre-training instance}
\end{figure}

\subsubsection{Fine-tuning Dataset: Single Log Generation} \label{sec:single-log-dataset}
We build a fine-tuning aimed at replicating what has been done in the training of LANCE by Mastropaolo \etal \cite{mastropaolo2022using}. We process each method $M$ having $n \geq 1$ log statements by removing from it one log statement (\ie leaving it with $n-1$ log statements). This allows to create a training pair $\langle M_s, M_t \rangle$ with $M_s$ representing the input provided to the model (\ie $M$ with one removed log statement) and  $M_t$ being the expected output (\ie $M$ in its original form, with all its log statements). This is the dataset used to train LANCE \cite{mastropaolo2022using} and it allows to train a model able, given a Java method as input, to inject in it one new log statement. 

For methods having $n > 1$ (\ie more than one log statement), we created $n$ pairs $\langle M_s, M_t \rangle$, each of them having one of the $n$ log statements removed (\ie different $M_s$). To ensure that after the log statement removal our instances still featured valid Java methods, we parsed each $M_s$ using JavaParser \cite{javaparser} and removed all pairs including an invalid $M_s$. 

We split the remaining pairs into training (80\%), validation (10\%) and test (10\%) set as reported in \tabref{tab:ds-summary-1}. Training and testing a T5 model on this dataset basically means performing a differentiated replication of LANCE on a much larger (+\textcolor{red}{XX\%} of instances) and more variegate (multiple logging libraries) dataset.

\subsubsection{Fine-tuning Dataset: Single Log Generation with IR} \label{sec:single-log-plus-IR}

In \approach, we want to combine DL and IR with the goal of boosting performance especially when it comes to the generation of meaningful log messages, being one of the weaknesses of LANCE. The main idea is to augment the input provided to the model (\ie $M_{s}$) with log messages belonging methods similar to $M_{s}$ which are featured in the fine-tuning training set. For each of the 244,588 $\langle M_s, M_t \rangle$ pairs in the fine-tuning dataset described in \secref{sec:single-log-dataset} (this includes training, validation, and test), we identify the $k$ most similar pairs in the training set. The similarity between two pairs is based on the similarity of their $M_s$ (\ie the method in which the log statement must be created) and it is computed using the Jaccard similarity \cite{hancock2004jaccard} index, based on the percentage of code tokens shared across the two methods. We then use these $k$ similar methods to extract from them example of log messages used in coding contexts which are similar to the $M_s$ at hand. Two clarifications are important here. First, independently if a given pair is in the training, validation, or test set, we extract its $k$ most similar pairs only from the training set. This is needed since, while predicting the log statement to inject, the training set must be the only knowledge available to the model (\ie the test set must be composed of previously unseen instances). Second, when computing the Jaccard similarity, we remove from the compared methods all log statements, since we want to identify similar ``coding contexts'' that may require similar log statements. We created three different fine-tuning datasets using different values of $k=\{1,3,5\}$ (thus, a higher/lower number of exemplar log messages provided as input to the model).

\figref{} shows an example of training instance for this fine-tuning dataset. The method on top represents the $M_{s}$ \java method in which a log statement must be injected (\ie the one highlighted in red). The method is enriched with the exemplar log messages that have been found in the $k=1$ most similar method shown in the bottom. Besides the log messages, we also provide T5 with the Jaccard similarity between the $M_{s}$ at hand (top of the figure in this case) and the method of the training set from which each log message has been extracted. This is just meant to represent an additional hint for T5 in terms of which exemplar message comes from the most similar coding context.

Note that the instances in this dataset are exactly the same of the one previously described to replicate LANCE (see \tabref{tab:ds-summary-1}). This allows a direct comparison in terms of performance which will provide information about the gain, if any, provided by the integration of the IR technique in the loop.

\subsubsection{Fine-tuning Dataset: Multi-log Injection with IR} \label{sec:multi-log-dataset}

The second limitation of LANCE \cite{mastropaolo2022using} we aim at addressing is the assumption that a Java method provided as input always require one new log statement to be injected. Also for this dataset, \approach exploits a combination of DL and IR, thus we follow a process similar to the one described in \secref{sec:single-log-plus-IR}, with the main difference being the number of log statements we ask the model to generate. In particular, given a method $M$ featuring $n$ log statements, we randomly select $y$ log statements to remove from it, with $1 \leq y \leq n$. This means that in this case we create pairs $\langle M_s, M_t \rangle$ in which $M_s$ lacks a ``random'' number of log statements that must be generated by the model to obtain the target method $M_t$. This makes the prediction task substantially more challenging as compared to the single-log injection scenario experimented in LANCE. Also in this case we parsed each $M_s$ using JavaParser \cite{javaparser} and removed all pairs including an invalid $M_s$. The remaining part of the process (\ie identifying the $k$ most similar pairs to inject examples of log messages) is exactly the same as the one described in \secref{sec:single-log-plus-IR}. \tabref{tab:ds-summary-1} shows the distribution of instances among the training, evaluation, and test set for this dataset as well.

%%%STOPPED HERE

\subsubsection{Fine-tuning Dataset: Deciding Whether Log Statements are Needed} \label{sec:predicting-dataset}

To build the dataset needed for predicting whether or not log statements are needed in a \java method, we use the same set of instances featuring the dataset built for the multi-log injection model. To this extent, we start from the original set of 244,588 \java methods having at least one log statement, then for each method, we chose an arbitrary number $k$ from 0 to $n$, where $n$ is the number of log statements in the method. Then, we randomly selected $k$ log statements out of the
method's $n$ logs. When $k$ is equal to 0 it means that no log statements have been removed and, thus, the input sequence and the target sequence are equal (\ie, the original \java method). After the log statements removal, we ensured that the input sequences still represented a valid \java code by using JavaParser \cite{javaparser} to parse the methods. All those instances throwing parsing exceptions have been removed from the datasets. 
Finally, for each instance, we replaced the target sequence with a binary choice either \texttt{Need} or \texttt{No need}. Instances labeled as \texttt{No need} correspond to the entries for which no log statements have been removed (\ie $k=0$) thus, need no further logs. On the other hand, the entries labeled as \texttt{Need} are those missing at least one log statement. It is possible that two methods differ only because they contain a different log statements, if said statements are removed then the methods could be considered equal. Thus, input sequences pointing to the same target sequences (\ie \texttt{Need}), have been removed to avoid duplicated entries in the dataset obtaining a training and a validation set feature by 190,974 instances and 23,725 respectively.

As for the test set, since it is not possible to ensure it represents the real distribution of Java methods that either need or do not need log statements, we decided to test the model's performance on three different test sets. Each test set contains a different distribution in terms of Java methods that need the injection of further logs or not. To this extent, we create three distributions:  (i) 50-50 (50\% of the instances require at least one additional log statement, the other 50\% do not), (ii) 75-25 (75\% of the methods want at least one additional log statement while the remaining 25\% do not) and, (iii) 25-75 (25\% of the instances need the injection of at least one log statements, while the other 75\% need none).
\tabref{tab:ds-summary-2} reports the number of instances contained in the above-mentioned datasets.

\begin{table*}[h!]
	\centering
	\caption{Num. of methods in the datasets used to predict the need for logs}
	\begin{tabular}{rcccccccc}
		\hline
		\multirow{2}{*}{\textit{\textbf{Dataset}}} & \multicolumn{2}{c}{\textbf{train}} & \textbf{} & \multicolumn{2}{c}{\textbf{eval}}  & \textbf{} & \multicolumn{2}{c}{\textbf{test}}  \\ \cline{2-3} \cline{5-6} \cline{8-9} 
		& \textbf{Need} & \textbf{No need}   & \textbf{} & \textbf{Need} & \textbf{No need}   & \textbf{} & \textbf{Need} & \textbf{No need}   \\ \hline
		\textit{Need4Log fine-tuning dataset (50-50)}         & 98,848        & 92,126             &           & 12,257        & 11,468             &           & 11,627        &  11,627            \\
		\textit{Need4Log fine-tuning dataset (75-25)}         & 98,848        & 92,126             &           & 12,257        & 11,468             &           & 12,159        &  4,053             \\
		\textit{Need4Log fine-tuning dataset (25-75)}         & 98,848        & 92,126             &           & 12,257        & 11,468             &           & 3,875         &  11,627            \\ \hline
	\end{tabular}
	\label{tab:ds-summary-2}
\end{table*}

\section{Training and Hyperparameter Tuning} \label{sec:training}
In this section we outline how the pre-training and the fine-tuning phase have been conducted to support the task of complete log statements generation (\ie injection of log statements) first and, the prediction of the need for log statements while working with \java methods. Afterwards, we also outline how we found the best-performing models for both tasks, exploiting two main strategies: (i) hyperparameter tuning  and (ii) early stopping.

The goal of the pre-training phase is to teach the model generalizable knowledge that is useful for the downstream task. To this extent, we want to teach the model how to handle the most recurrent patter of \java. This is achieved by leveraging a specific pre-training objective, namely, MLM (Masked-Language-Modeling). Said objective consists in teaching the model to predict missing or masked tokens in the input. We processed the methods to obtain an input sequence in which 15\% of random tokens have been masked as depicted in \figref{fig:pre-training}.
We pre-train the T5 model for 500k steps on a dataset of 12,671,475 instances using Google Colab's 2x2, 8 cores TPU topology with a batch size of 128 and a maximum sequence length of 512.
To this extent, Mastropaolo \etal \cite{mastropaolo2022using} found out that among the different pre-training strategies they experimented with, the one above described (\ie denoising-task) led to the best results. Thus, we opted for pre-training the T5 model on a bigger dataset using the same strategy to build knowledge that can be re-used while fine-tuning the model.

Once the model has been pre-trained, we can specialize it to solve a specific problem using a fine-tuning task. The objective of LANCE \cite{mastropaolo2022using} was to investigate the extent to which  T5 represented a doable approach in generating complete log statements. More specifically, its ability to generate and inject a single log statement in a \java method. Thus, given a \java method from which we removed one log statement at a time  (\secref{sec:single-log-dataset}), the model has to predict in which position to inject the log statement, what is the correct log level and has to generate a meaningful log message as well. Thus, as output, we expect the model to return the same \java method with, however, the injected complete log statement. 

Similarly, we would expect that while fine-tuning the pre-trained model on the augmented dataset (\ie  \textit{Single-Log Context-Aware fine-tuning dataset}), the contextual information (\ie log messages alongside the Jaccard similarity values) added to the input sequences can help the model in synthesizing more accurate log messages.

Concerning the hyperparameters tuning phase, as discussed by Mastropaolo \etal \cite{mastropaolo2021studying} in their seminal work introducing T5 for code-related tasks, we did not tune the T5 model hyperparameters during the pre-training phase, because such a phase is task-agnostic, and therefore it would provide limited benefits. Instead, we conduct such a phase, using the same strategy that was used when fine-tuning LANCE  \cite{mastropaolo2022using}. Thus, we experiment with four different learning rate scheduler: (i) \textit{Constant Learning Rate} (C-LR): the learning rate is fixed during the whole training (we use $LR = 0.001$, \ie the value used in the original paper \cite{raffel2019exploring}); (ii) \textit{Inverse Square Root Learning Rate} (ISR-LR): the learning rate decays as the inverse square root of the training step (the same used for pre-training by Raffel \etal); (iii) \textit{Slanted Triangular Learning Rate \cite{howard2018universal}} (ST-LR): the learning rate first linearly increases and then linearly decays to the starting learning rate;  (iv) \textit{Polynomial Decay Learning Rate} (PD-LR): the learning rate decays polynomially from an initial value to an ending value in the given decay steps.
Table \ref{tab:learning-rates} reports all the parameters used for each scheduling strategy as evaluated in \cite{mastropaolo2022using}.

\begin{table}[h]
	\centering
	\begin{tabular}{ll}
		\hline
		\textbf{Learning Rate Type} & \textbf{Parameters}               \\ \hline
		Constant                     & \textit{LR = 0.001}               \\
		Inverse Square Root         & \textit{LR\textsubscript{starting} = 0.01}  \\
		& \textit{Warmup = 10,000}          \\
		Slanted Triangular          & \textit{LR\textsubscript{starting} = 0.001} \\
		& \textit{LR\textsubscript{max} = 0.01}       \\
		& \textit{Ratio = 32}               \\
		& \textit{Cut = 0.1}                \\
		Polynomial Decay            & \textit{LR\textsubscript{starting} = 0.01}  \\
		& \textit{LR\textsubscript{end} = 0.001}      \\
		& \textit{Power = 0.5}              \\ \hline
	\end{tabular}
	\vspace{0.2cm}
	\caption{Configurations for all the learning rate strategies used in this study}
	\label{tab:learning-rates}
\end{table}


\begin{table*}[h!]
	\centering
	\caption{T5 hyperparameter tuning results while supporting the injection of single and multi log statements. The strategy leading to the best results is reported in bold.}
	\begin{tabular}{lrrrr}
		\toprule
		\textbf{Experiment}                  																		& \textbf{C-LR}              & \textbf{ST-LR}      & \textbf{ISQ-LR}        & \textbf{PD-LR} \\
		\midrule
		\textit{Single-Log Context-Aware fine-tuning dataset} ($K=1$)                         &   24.63\%                & 25.92\%    		           & \textbf{26.55\%}           &  26.36\%         \\
		\textit{Single-Log Context-Aware fine-tuning dataset} ($K=3$)                        &   26.25\%                & 26.04\%    		           & \textbf{26.68\%}          &  26.33\%         \\
		\textit{Single-Log Context-Aware fine-tuning dataset} ($K=5$)                         &  26.24\%                & 25.69\%    		           & \textbf{26.78\%}           &  26.33\%         \\
		\bottomrule
		\textit{Multi-Log fine-tuning dataset}                       										&   21.77\%                & 20.88\%    		           & \textbf{21.84}\%           &  21.57\%         \\
		\bottomrule
		\textit{Multi-Log Context-Aware fine-tuning dataset} ($K=1$)                         &   22.62\%                & 22.19\%    		           & \textbf{22.79\%}           &  22.76\%         \\
		\textit{Multi-Log Context-Aware fine-tuning dataset} ($K=3$)                        &   22.64\%                & 22.28\%    		           & \textbf{23.05\%}          &  22.59\%         \\
		\textit{Multi-Log Context-Aware fine-tuning dataset} ($K=5$)                         &   22.71\%                & 22.14\%    		           & \textbf{22.78\%}           &  22.51\%         \\
		\bottomrule
	\end{tabular}
	
	\label{tab:hp-results}
\end{table*}

Since we use software-specific corpora to pre-train and fine-tune the model we need a new vocabulary that include the \java tokens featuring our datasets. For this reason, we trained a new tokenizer (\emph{i.e.}, a SentencePiece model \cite{kudo2018sentencepiece}) on 1M Java methods and 712,634 English sentences coming from the C4 dataset \cite{raffel2019exploring}. Similarly to what has been done by Mastropaolo \etal \cite{mastropaolo2022using}, we included English sentences to deal with complex log messages and we set the size of the resulting vocabulary to 32k word-pieces.

\subsubsection{Single Log Statements Injection}

We must anticipate that we do not perform any hyperparameters tuning while replicating LANCE on our \textit{Single-Log fine-tuning dataset}, but instead, we train our T5 model supporting the complete injection of log statements on our novel dataset, using the best configuration of hyperparameters found by Mastropaolo \etal \cite{mastropaolo2022using}. In this regard, we used the Polynomial Decay Learning Rate (PD-LR), whose parameters are reported in \tabref{tab:learning-rates}. Afterward, we used the validation datasets to gauge the model accuracy every 10,000 steps and perform early stopping, choosing a delta of 0.01, and patience of 5. In doing so, we were able to select the model after 240k steps that most likely achieved the best results in terms of generalizability while avoiding overfitting.

In contrast, to find the best-performing model when fine-tuning T5 on the \textit{Single-Log Context-Aware fine-tuning dataset}, we train 12 different models (\ie four models for each experimented $K$ value) for a total of 100k steps using a sequence length of 1024 (both input and output). In addition, we doubled the number of tokens to manage the extra contextual information (\ie log messages and Jaccard similarities) added to the input sequences. At the end of such a phase, for each $K=\{1,3,5\}$, we select the model that achieves the highest number of correct predictions (\ie cases in which the predicted output sequences are equal to the oracle).

From the achieved results reported in \tabref{tab:hp-results}, it emerges a slight gain in terms of correct predictions when using the ISR-LR scheduler across the different values of $K$. Thus, we use such a scheduler to fine-tune the models on the	\textit{Single-Log Context-Aware fine-tuning dataset}. In detail, when $K=1$, we fine-tune the T5 model for 120k steps, after which we did not find improvements in going further, resulting in the activation of the early stopping strategy. When the number of methods from which we retrieve the log messages increases (\ie $K=3$ and $K=5$), the model reaches convergence after 120k steps for $K=3$, and after 190k steps for $K=5$.

\subsubsection{Multi Log Statements Injection} \label{sec:multi-injection-model}

Concerning the multi-log injection task, we perform hyperparameters tuning for both configurations (\ie augmented data and not). This resulted in the fine-tuning of 20 T5 models achieving the results reported in \tabref{tab:hp-results} (see rows from 4 to 7). In particular, from the performances, we report in \tabref{tab:hp-results} become evident a clear trend in which the ISQ-LR scheduler outperforms all the other strategies in the set of experiments we carried out. To this extent, each model has been trained using the ISQ-LR scheduler using an early stropping strategy that stops the training if no improvements in terms of correct predictions are found after 5 evaluating the model for 5 consecutive checkpoints (a new checkpoint is saved after 10k steps of fine-tuning).
The best-performing model generating log statements using the non-augmented inputs converges after \textcolor{red}{200k} steps.
On the other hand, the models that have been fine-tuned using the augmented data (\ie log messages and the Jaccard similarities of the methods), reach convergence after 270k steps for $K=1$. In contrast, when the number of methods from which we retrieve the log messages increase (\ie $K=3$ and $k=5$), both models reach convergence after 230k steps.

\subsubsection{Predicting The Need for Log Statements}
As for predicting the need for log statements, we decided to investigate the extent to which the T5 model we pre-trained for supporting the task of complete log statements generation, can further be fine-tuned on the dataset we introduced in \secref{sec:predicting-dataset} predicting whether log statements are needed or not.
Therefore to study the feasibility of a large pre-trained language model of code (\ie T5) in supporting such a task, we first found the best configurations of hyperparameters. Later, to tame overfitting we adopt the discussed early stopping strategy, evaluating the model on the validation set every 10k steps and stopping if there are no gains in terms of accuracy after 5 consecutive rounds of evaluation.
We conducted the search of the learning strategy maximizing the accuracy, fine-tuning four different models (for 100k steps), each one of which features one of the scheduler reported in \tabref{tab:learning-rates} . From the achieved results reported in \tabref{tab:need4log-hp}, the scheduler that achieves the best results is the PD-LR. Thus, we selected such a learning strategy to fine-tune the classifier model for 30k steps, after which the early stopping strategy returns the best checkpoint.

Once the best-performing T5 based classifier has been fine-tuned, it can be queued by a model able to inject log statements into \java methods (\secref{sec:multi-injection-model}).

  \begin{table}[h!]
	\centering
	\caption{T5 hyperparameters tuning results while supporting the task of predicting the need for log statements}
	\begin{tabular}{rrrrr}
		\hline
		\textbf{Experiment}        & \textbf{C-LR} & \textbf{ST-LR} & \textbf{ISQ-LR}  & \textbf{PD-LR}  \\ \hline
		\textit{Need4Log} & 96.58\%       & 96.56\%        & 96.59\%          & \textbf{96.62\%}\\ \hline
	\end{tabular}
	\label{tab:need4log-hp}
\end{table}

\subsubsection{Generating Predictions}
Once the T5 model has been pre-trained and fine-tuned to support the task of complete log statements generation and prediction of the need for logs, it can generate predictions using different decoding strategies. In this regard, we leveraged the same schema adopted by Mastropaolo \etal \cite{mastropaolo2022using}, implementing the greedy decoding strategy. Such a decoding schema generates the final predictions by selecting at each decoding step the token with the highest probability of appearing in a specific position. In doing so, a single prediction (\ie the one maximizing the likelihood of among all the produced tokens) is generated for the input sequence we give as input to the model.





% !TEX root = main.tex
%%%%%%%%%%%%%%%%%%%%%%%%%%%%%%%%%%%%%%%%
%%%%%%%%%%%%%%%%%%%%%%%%%%%%%%%%%%%%%%%%
\section{Study Design} \label{sec:design}
%%%%%%%%%%%%%%%%%%%%%%%%%%%%%%%%%%%%%%%%
%%%%%%%%%%%%%%%%%%%%%%%%%%%%%%%%%%%%%%%%

The \emph{goal} of our study is to evaluate the performance of \approach in supporting logging activities in \java methods. We focus on three scenarios: single log injection, in which we compare with the state-of-the-art approach LANCE \cite{mastropaolo2022using}; multi-log injection; and deciding wether log statements are needed or not in a given \java method. The context is represented by the test datasets reported in \tabref{tab:ds-summary-1} (single and multi-log injection) and \tabref{tab:ds-summary-2} (deciding whether logging is needed).

We aim at answering the following research questions:

\begin{itemize}[itemindent=0.3cm]

\item[\textbf{RQ$_1$:}]\textit{To what extent is \approach able to correctly inject a single complete logging statement in Java methods?} RQ$_1$ mirrors the study performed by Mastropaolo \etal~\cite{mastropaolo2022using} when presenting LANCE. We experiment \approach in the same scenario presented in \cite{mastropaolo2022using}: The injection of a single log statement in a given Java method. We compare the performance of \approach with that of LANCE when training and testing them on the same dataset. 

\item[\textbf{RQ$_2$:}]\textit{To what extent is \approach able to correctly inject multiple log statements when needed?} RQ$_2$ tests \approach in the more challenging scenario of injecting from 1 to $n$ log statements in a \java method, as needed.

\item[\textbf{RQ$_3$:}]\textit{To what extent is \approach able to properly decide when to inject log statements?} RQ$_3$ analyzes the accuracy of \approach in predicting whether or not log statements are needed in a given \java method, a problem that was oversaw in the work presenting LANCE \cite{mastropaolo2022using}.

\end{itemize}

\subsection{Data Collection and Analysis}

To answer RQ$_1$ we run both \approach and LANCE against the test set described in \tabref{tab:ds-summary-1} for the single log generation task. The only difference is that LANCE has been trained on the dataset not featuring the exemplar log messages added through IR (row \emph{Fine-tuning: Single Log Generation} in \tabref{tab:ds-summary-1}), while \approach exploits this information (row \emph{Fine-tuning: Single Log Generation with IR} in \tabref{tab:ds-summary-1}). However, the training and test instances are exactly the same, allowing for a direct comparison. We assess the performance of the two techniques using the same evaluation schema of Mastropaolo \etal~\cite{mastropaolo2022using}. In particular, we contrast the predictions generated by the two models against the expected output (\ie the \java method provided as input with the addition of the correct log statement). Note that generating and injecting a log statement  (\eg \texttt{LoggerUtil.debug("execution ok")}) involves correctly predicting several information: (i) the name of the variable used for the logging (\ie \texttt{LoggerUtil}); (ii) the log level (\ie \texttt{debug}); (iii) the log message (\ie \texttt{"execution ok"}); and (iv) the position in the method in which the log statement must be injected. Thus, when a prediction is generated, three scenarios are possible:

\textbf{Correct prediction:} A prediction that correctly captures all above-described information, \ie it matches the name used for the variable, the log level, message, and position as written by the original developers.

\textbf{Partially correct prediction:} A prediction that correctly captures a subset of the needed information (\eg it correctly generates the log statement but injects it in the wrong position).

\textbf{Wrong prediction:} None of the above-described information is correctly predicted.

We answer RQ$_1$ through the following combination of quantitative and qualitative analysis. On the quantitative side, we report for both \approach and LANCE the percentage of correct, partially correct, and wrong predictions. For the partially correct, we report the percentage of cases in which each of the ``log statement components'' (\ie variable name, log level, log message, and log position) has been correctly predicted. As for the percentage of correct and partially correct predictions, we pairwise compare them among the experimented techniques, using the McNemar's test \cite{mcnemar}, which is a proportion test suitable to pairwise compare dichotomous results of two different treatments. We complement the McNemar's test with the Odds Ratio (OR) effect size. We use the Holm's correction procedure \cite{Holm1979a} to account for multiple comparisons.

Concerning the quality of the log messages generated by the two techniques, looking for exact matches (\ie cases in which the generated log message is identical to the one written by developers) is quite limitative considering that a prediction including a message different but semantically equivalent to the target one could still be valuable. For this reason, we also compute the following four metrics used in Natural Language Processing (NLP) for the assessment of automatically generated text:

\textbf{BLEU}~\cite{papineni2002bleu} assesses the quality of the automatically generated text in terms of $n$-grams overlap with respect to the target text. The BLEU score ranges between 0 (the sequences are completely different) and 1 (the sequences are identical) and can be computed considering four different values of $n$ (\ie BLEU-\{1, 2, 3, 4\}). Besides these four variants, we also compute their geometric mean (\ie BLEU-A).

\textbf{METEOR}~\cite{meteor} is a metric based on the harmonic mean of unigram precision and recall. Compared to BLEU, METEOR uses stemming and synonyms matching to better reflect the human perception of sentences with similar meanings. Values range from 0 to 1, with 1 being a perfect match.

\textbf{ROUGE}~\cite{lin2004rouge} is a set of metrics focusing on automatic summarization tasks. We use the ROUGE-LCS (Longest Common Subsequence) variant which returns three values: the recall computed as \textit{LCS(X,Y)/length(X)}, the precision computed as \textit{LCS(X,Y)/length(Y)}, and the F-measure computed as the harmonic mean of recall and precision, where \textit{X} and \textit{Y} represent two sequences of tokens.

\textbf{LEVENSHTEIN Distance}~\cite{levenshtein1966} provides an indication of the percentage of words that must be changed in the synthesized log message to match the target log message. This is accomplished by computing the normalized token-level Levenshtein distance \cite{levenshtein1966} (NTLev) between the predicted log message and the target one. Such a metric can act as a proxy to estimate the effort required to a developer in fixing a non-perfect log message suggested by the model.

% Un-comment if we have space
%$$
%NTLev(LM_p, LM_t) = \frac{\mathit{TLev}(LM_p, LM_t)}{\max({\{}|LM_p|, |LM_t|\})}
%$$
%
%\noindent with $\mathit{TLev}$ representing the token-level Levenshtein distance between the two log messages. 



We also statistically compare the distribution of the BLEU-4 (computed at sentence level), METEOR, ROUGE, and LEVENSHTEIN distance related to the predictions generated by \approach and LANCE. We assume a significance level of 95\% and use the Wilcoxon signed-rank test \cite{wilcoxon}, adjusting $p$-values using the Holm's correction \cite{Holm1979a}. The  Cliff's Delta ($d$) is used as effect size \cite{Gris2005a} and it is considered: negligible for $|d| < 0.10$, small for $0.10 \le |d| < 0.33$, medium for $0.33 \le |d| < 0.474$, and large for $|d| \ge 0.474$ \cite{Gris2005a}.

On the qualitative side, we manually inspected 300 of the partially correct predictions generated by both techniques and having all information but the log message correctly predicted. The goal of the inspection was to verify whether the generated log message, while different from the target one, was semantically equivalent to it. To this aim, two of the authors independently inspected all 600 log messages (300 for each approach), with $\sim$11\% (70) arisen conflicts being solved by a third author. We report the percentage of ``wrong'' log messages generated by both techniques classified as semantically equivalent to the target one.

To answer RQ$_2$ and evaluate the extent to which \approach is able to correctly inject multiple log statements, we run \approach against the test set reported in \tabref{tab:ds-summary-1} (see row \emph{Fine-tuning: Multi-log Injection with IR}). We then report the percentage of correct predictions generated by the approach (\ie methods for which all $n$ log statements that \approach was supposed to generate and inject have been correctly predicted). In this case we do not compute the partially correct predictions since, if a prediction is not completely correct, it is not possible to match the generated log statements with the target ones to compare them. To make this concept more clear, consider the case in which \approach was asked to generate two log statements $s_1$ and $s_2$ but it only injects one statement $s_i$, being different from both $s_1$ and $s_2$. We cannot know whether $s_i$ should be compared with $s_1$ or with $s_2$ to assess the percentage of partially correct predictions in terms of \eg log level. For this reason, we only focus on the predictions being 100\% correct (\ie the output method is identical to the target one). 

To answer RQ$_3$, we run \approach against the test sets presented in \tabref{tab:ds-summary-2}, reporting the confusion matrix of the generated predictions and the corresponding accuracy, recall, and precision. We compare these results with those of: (i) an \emph{optimistic} classifier always predicting \emph{true} (\ie the method is in need for log statements); (ii) a \emph{pessimistic} classifier always predicting \emph{false} (\ie no need for log statements); and (iii) a random classifier, randomly predicting \emph{true} or \emph{false} for each input instance. We use the same statistical analysis described for RQ$_1$ to compare \approach with the baselines.
% !TEX root = main.tex
%%%%%%%%%%%%%%%%%%%%%%%%%%%%%%%%%%%%%%%%
%%%%%%%%%%%%%%%%%%%%%%%%%%%%%%%%%%%%%%%%
\section{Results Discussion} \label{sec:results}
%%%%%%%%%%%%%%%%%%%%%%%%%%%%%%%%%%%%%%%%
%%%%%%%%%%%%%%%%%%%%%%%%%%%%%%%%%%%%%%%%
% !TEX root = main.tex
%%%%%%%%%%%%%%%%%%%%%%%%%%%%%%%%%%%%%%%%
%%%%%%%%%%%%%%%%%%%%%%%%%%%%%%%%%%%%%%%%
\section{Threats to Validity} \label{sec:threats}
%%%%%%%%%%%%%%%%%%%%%%%%%%%%%%%%%%%%%%%%
%%%%%%%%%%%%%%%%%%%%%%%%%%%%%%%%%%%%%%%%

\textbf{Construct validity.} The building of our fine-tuning datasets rely on the assumption that the exploited code instances, as written by developers, represent the ``correct'' predictions that the models should generate. This is especially true for the classifier aimed at predicting whether log statements are needed. For example, the instances that we labeled as ``\emph{not needing log statements}'' are methods featuring $n \geq 1$ log statements from which we did not remove any log statement. Thus, we assume that these methods need exactly $n$ log statements (\ie the ones injected by the developers), not one more. This is a strong assumption, as confirmed by the examples in \figref{fig:no-need}. Still, using the code written by developers as oracle is a popular practice in DL for SE \cite{tufano2022using, Tufano:tosem2019, tufano-mutants, watson2020learning, tufano2022generating}.


\textbf{Internal validity.}  We performed the same hyperparameter tuning we proposed when introducing the T5 model to support code-related tasks \cite{mastropaolo2021studying}, while relying on the best architecture identified by Raffel \etal \cite{raffel2019exploring} for the other parameters. We acknowledge that additional tuning can result in improved performance.


\textbf{External validity.} Our research questions have been answered using a dataset being 3.6 times larger as compared to the dataset we originally used when proposing \cite{mastropaolo2022using}. Also, the new dataset is more variegated, featuring projects using different build systems (as compared to the Maven-only policy we relied in \cite{mastropaolo2022using}) and having dependencies towards different logging libraries (differently from the original Log4j-only policy we end up using in \cite{mastropaolo2022using}). Still, we do not claim generalizability of our findings for different populations of projects, especially those written in other programming languages.
% !TEX root = main.tex
%%%%%%%%%%%%%%%%%%%%%%%%%%%%%%%%%%%%%%%%
%%%%%%%%%%%%%%%%%%%%%%%%%%%%%%%%%%%%%%%%
\section{Related Work} \label{sec:related}
%%%%%%%%%%%%%%%%%%%%%%%%%%%%%%%%%%%%%%%%
%%%%%%%%%%%%%%%%%%%%%%%%%%%%%%%%%%%%%%%%

In this work, we focus our discussion on studies aimed at (i) automating logging activities and (ii) combining deep learning and information retrieval to automate code-related activities. Due to space constraints, we point the reader to the literature review by Watson \etal~\cite{watsonSytematicLiterature2020} regarding the use of deep learning to automate different software engineering tasks.

\subsection{Automating Logging Activities}

Log analysis is a popular practice that support developers in software maintenance tasks such as software testing~\cite{chen2018automated,chen2019experience}, debugging~\cite{satyanarayanan1992transparent}, diagnosis~\cite{zhou2019latent,yuan2012improving}, and monitoring~\cite{hasselbring2020kieker,harty2021logging}. Nonetheless, augmenting source code with correct log statements is a non-trivial manual activity that require developer's experience~\cite{li2020qualitative,yuan2012characterizing}. For this reason, researchers started investigating how to support developers with automatic log injection.

Zhu \etal~\cite{zhu2015learning} conducted a pioneer work to design \textsc{LogAdvisor}. It is a tool to recommend where to inject log statements in the source code. The authors evaluated \textsc{LogAdvisor} on two Microsoft systems and two open-source projects. They reported an accuracy of 60\% when the tool is triggered to inject log statements on pieces of code without log statements.

Yao \etal~\cite{yao2018log4perf} have focused on a similar sub-problem: The proposed approach automatically suggests log statements in code developed to monitor the CPU usage of web-based systems. The results demonstrated a boost in such developers' activities.

Mizouchi \etal \cite{mizouchi2019padla}, moved ahead, and designed \textsc{PADLA}, an extension for the Apache Log4j framework to automatically change log levels of existent log statements. \textsc{PADLA} aims to adjust the log level to optimize the amount of the information logged at runtime and allow a responsive anomaly analysis.

Li \etal~\cite{li2020shall} introduced deep learning to generate logging location recommendations. Their tool reports a 80\% accuracy in suggesting logging locations using within-project training, with slightly worst results (67\%) in a cross-project setting.

Recently, Li \etal \cite{li2021deeplv} proposed also \textsc{DeepLV}. It is a tool based on deep learning to recommend the level of existing log statements in Java methods. \textsc{DeepLV} by aggregating both syntactic and semantic information of the source code showed its superiority with respect to the state-of-the-art tools.

Lastly, Mastropaolo \etal~\cite{mastropaolo2022using} introduced \textsc{LANCE}, a tool to inject log statements enriched with log levels and messages automatically. The proposed tool achieves outstanding performance when solving the first two tasks (65\% and 66\%, respectively, for log position and log levels) while hobbling on the most complex challenge: Generating natural language log messages. Moreover, their tool generates a single log message per method. With our study, we build on top of the above research, and experiment with an augmented model that outperforms \textsc{LANCE} by injecting multiple log statements. 

\subsection{Combining Deep Learning and IR to Automate Code Related Tasks}

Although deep learning showed great potential in solving various software engineering tasks~\cite{watsonSytematicLiterature2020}, it may still struggle with real-world challenges such as generating natural language messages for non-trivial log statements~\cite{mastropaolo2022using}. To overcome such kind of limitations, researchers aimed at leveraging alternative solutions such as combining deep learning and information retrieval to better reuse recorded data..

Yu \etal~\cite{yu2022automated} proposed an innovative approach that combines deep learning and information retrieval techniques. The authors augmented \textsc{ATLAS} to re-use available snippet of source code when the performance of their deep learning model shows a drop. The tool selects an existent code assertion through information retrieval and suggests reusing it. The key concept behind information retrieval is to fetch from a pool of available data the most similar subject. Researchers have leveraged various similarity coefficients in information retrieval such as the Jaccard~\cite{tanimoto1958elementary} similarity. It is a similarity
coefficient used to estimate the similarity between two sets of data based on their overlapping and unique items.


% !TEX root = main.tex
%%%%%%%%%%%%%%%%%%%%%%%%%%%%%%%%%%%%%%%%
%%%%%%%%%%%%%%%%%%%%%%%%%%%%%%%%%%%%%%%%
\section{Conclusions and Future Work} \label{sec:conclusions}
%%%%%%%%%%%%%%%%%%%%%%%%%%%%%%%%%%%%%%%%
%%%%%%%%%%%%%%%%%%%%%%%%%%%%%%%%%%%%%%%%

We started by discussing the limitations of LANCE \cite{mastropaolo2022using}, the state-of-the-art approach for the generation of complete log statements. LANCE always assumes that a \emph{single} log statement \emph{must} be injected in a method provided as input. It is a strong assumption considering that a method may not need logs or may need more than one log statement. Thus, we presented \approach, an extension of LANCE able to partially address these two limitations, making a further step ahead in the automation of logging activities. Moreover, we experimented a combination of DL and IR with the goal of improving the generation of meaningful log messages achieving, however, only limited improvements over LANCE. 

We are working on implementing \approach as a tool to be deployed to developers. This is the next step needed to perform \emph{in vivo} studies, thus better understanding the main weaknesses of current DL-based log generation. 

%\section*{Acknowledgment}
%This project has received funding from the European Research Council (ERC) under the European Union's Horizon 2020 research and innovation programme (grant agreement No. 851720). 

\bibliography{main}
\bibliographystyle{IEEEtran}

\end{document}
