% !TEX root = main.tex
%%%%%%%%%%%%%%%%%%%%%%%%%%%%%%%%%%%%%%%%
%%%%%%%%%%%%%%%%%%%%%%%%%%%%%%%%%%%%%%%%
\section{Conclusions and Future Work} \label{sec:conclusions}
%%%%%%%%%%%%%%%%%%%%%%%%%%%%%%%%%%%%%%%%
%%%%%%%%%%%%%%%%%%%%%%%%%%%%%%%%%%%%%%%%

We started by discussing the limitations of LANCE \cite{mastropaolo2022using}, the state-of-the-art approach for the generation of complete log statements. LANCE always assumes that a \emph{single} log statement \emph{must} be injected in a method provided as input. It is a strong assumption considering that a method may not need logs or may need more than one log statement. Thus, we presented \approach, an extension of LANCE able to partially address these two limitations, making a further step ahead in the automation of logging activities. Moreover, we experimented a combination of DL and IR with the goal of improving the generation of meaningful log messages achieving, however, only limited improvements over LANCE. 

We are working on implementing \approach as a tool to be deployed to developers. This is the next step needed to perform \emph{in vivo} studies, thus better understanding the main weaknesses of current DL-based log generation. 